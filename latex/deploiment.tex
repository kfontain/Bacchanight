- Font :
    field_font
    File - File
    bacchanight/font

- Couleur 1 et 2 :
    field_couleur_principale field_couleur_secondaire
    Text - Text field
    size : 7
    #FFFFFF

- Images :
    field_image_fond field_image_description
    Image - Image
    .gif .png .jpg .jpeg (default)
    bacchanight/img

- Description :
    field_description
    Long Text - Text area
    balises : <strong> <br> <p>


- Programme :
    field_program
    File - File
    .txt (default)
    bacchanight/txt

- Image programme :
    field_image_program
    Image - Image
    .gif .png .jpg .jpeg (default)
    bacchanight/img

- En continu :
    field_continu
    comme Programme

- Galeries :
    field_galeries
    Image - Image
    not required
    0 à 10 images
    .gif .png .jpg .jpeg (default)
    bacchanight/img

- Captcha :
    field_captcha_key
    Text - Text field
    not required
    size : 255

- Texte partenaire :
    field_texte_partenaire
    comme Description

- Image partenaire :
    field_image_partenaire

- Date et heure :
    Texte - Text field
    field_date
    date size : 30
    field_time
    time size : 25

- Téléphone :
    field_phone
    Texte
    size : 20

- Mail :
    field_mail
    Texte
    size : 255

- Facebook & co :
    field_facebook field_insta field_twitter
    Texte
    size : 255 255 255

- Copyright :
    field_copyright
    Texte
    size : 255

- Image BX :
    field_image_bordeaux
    Image
    b/img

- Image MBA :
    field_image_mba
    Image
    b/img

- Google analytics :
    field_ga
    texte
    not required
    size : 20

DEPLOIMENT 1 :
    - Créer un nouveau type de content avec les champs comme dessus.
    - Créer un nouveau contenu du type créer précédemment.
    - Copier le fichier dans <depot_drupal>/themes/<mon_theme>/templates.
    - Renommer le fichier <page--node--[NUMERO DU NOEUD].tpl.php
    - CLEAR ALL CACHES.

DEPLOIMENT 2 :
    - Créer un nouveau type de content avec les champs comme dessus.
    - Créer un nouveau contenu du type créer précédemment.
    - Copier la fonction donnée dans le fichier <depot_drupal>/themes/<mon_theme>/template.php sauf si elle existe déjà.
    - Cette fonction permet de rajouter comme nom de template possible page--[NOM TYPE CONTENT].tpl.php.
    - Copier le fichier dans <depot_drupal>/themes/<mon_theme>/templates.
    - Renommer le fichier <page--[NOM TYPE CONTENT].tpl.php
    - CLEAR ALL CACHES.
